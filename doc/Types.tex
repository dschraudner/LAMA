\title{LAMA Types}

\documentclass[12pt]{article}

\usepackage[utf8]{inputenc}
\usepackage[ngerman]{babel}

\usepackage{amsmath}
\usepackage{amssymb}
\usepackage{bussproofs}

\newcommand{\syn}[1]{\texttt{{#1}}}
\newcommand{\ty}[1]{\textbf{#1}}

\begin{document}
\maketitle

\section{...}

Typisierung eines LAMA-Programms bestehend aus:
Typdeklarationen ($T$), Konstantendeklarationen ($C$), Variablendeklarationen ($D$),
Datenfluss ($F$), Anfangszustand ($I$), Assertion ($A$) und Invarianten ($P$).

%<*typerules>

\begin{figure}[H]

%<*progrule>
\begin{prooftree}
  \alwaysNoLine
  \AxiomC{$ \vdash T \therefore \Sigma$}
  \UnaryInfC{$\Sigma,\Delta,\Gamma \vdash F : \ty{ok}$}
  \UnaryInfC{$\Sigma,\Delta,\Gamma \vdash A : \ty{ok}$}

    \AxiomC{$\Sigma \vdash C \therefore \Delta$}
    \UnaryInfC{$\Sigma,\Delta,\Gamma \vdash I : \ty{ok}$}
    \UnaryInfC{$\Sigma,\Delta,\Gamma \vdash P : \ty{ok}$}

      \AxiomC{$\Sigma \vdash D \therefore \Gamma$}
      \UnaryInfC{\phantom{$\Delta,$}} % empty premisses
      \UnaryInfC{\phantom{$\emptyset,\Sigma \vdash D
          \therefore \Delta_2$}} % force correct line length

  \alwaysSingleLine
  \RightLabel{(program)}
  \TrinaryInfC{$\vdash T \; C \; D \; F \; I \; A \; P : \ty{ok}$}
\end{prooftree}

%</progrule>

%<*typedefrules>
% typdef
\begin{prooftree}
\AxiomC{$\vdash T_1 \therefore \Sigma_1, \dots,
  T_n \therefore \Sigma_n$}
\RightLabel{(typedef)}
\UnaryInfC{$\vdash \syn{typedef } T_1 \ldots T_n \syn{ ;}
  \therefore \bigcup_{i=1}^n\Sigma_i$}  
\end{prooftree}

% enum
\begin{prooftree}
\AxiomC{}
\RightLabel{(enum)}
\UnaryInfC{$\vdash \syn{enum } x \syn{ = \{} c_1, \ldots, c_n  \syn{\}}
  \therefore \{(x, \{c_i \colon \ast \Rightarrow x\})\}$}
\end{prooftree}

%</typedefrules>

% constants
\begin{prooftree}
\AxiomC{$\Sigma \vdash c_1 : A_1 \ldots c_n : A_n$}
\RightLabel{(constants)}
\UnaryInfC{$\Sigma \vdash \syn{constants }
  x_1 \syn{ = } c_1 \syn{;} \ldots \; x_n \syn{ = } c_n \syn{;}
  \therefore \{(x_1, A_1), \ldots, (x_n, A_n)\} $}
\end{prooftree}

%<*invariantrule>
% invariant
\begin{prooftree}
\AxiomC{$\Sigma,\Delta,\Gamma \vdash e : \ty{bool}$}
\RightLabel{(invariant)}
\UnaryInfC{$\Sigma,\Delta,\Gamma \vdash
  \syn{invariant } e \syn{ ;} : \ty{ok}$}
\end{prooftree}
%</invariantrule>

\caption{Typisierung von LAMA-Programmen}
\label{fig:lama_program_types}
\end{figure}

%%% declarations
\begin{figure}[H]

% declarations
\begin{prooftree}
\AxiomC{$\Sigma, \Delta \vdash N \therefore \Gamma_1$}
\AxiomC{$\Sigma \vdash L \therefore \Gamma_2$}
\AxiomC{$\Sigma \vdash S \therefore \Gamma_3$}
\RightLabel{(declarations)}
\TrinaryInfC{$\Sigma,\Delta \vdash N \; L \; S
  \therefore \Gamma_1 \cup \Gamma_2 \cup \Gamma_3 $}
\end{prooftree}

% nodes
\begin{prooftree}
\AxiomC{$\Sigma, \Delta \vdash N_1 : A_1, \, \ldots, \, N_n : A_n$}
\RightLabel{(nodes)}
\UnaryInfC{$\Sigma, \Delta \vdash \syn{nodes } N_1 \ldots N_n
  \therefore \{(N_1,A_1), \ldots, (N_n, A_n)\}$}  
\end{prooftree}

% local
\begin{prooftree}
\AxiomC{$A_1 = \Sigma(t_1), \, \ldots, \, A_n = \Sigma(t_n)$}
\RightLabel{(local)}
\UnaryInfC{$\Sigma \vdash \syn{local }
  x_1 \syn{ : } t_1 \syn{;} \; \ldots \; x_n \syn{ : } t_n \syn{;}
  \therefore \{(x_1,A_1), \ldots, (x_n, A_n)\}$}  
\end{prooftree}

% state
\begin{prooftree}
\AxiomC{$A_1 = \Sigma(t_1), \, \ldots, \, A_n = \Sigma(t_n)$}
\RightLabel{(state)}
\UnaryInfC{$\Sigma \vdash \syn{state }
  x_1 \syn{ : } t_1 \syn{;} \; \ldots \; x_n \syn{ : } t_n \syn{;}
  \therefore \{(x_1,A_1), \ldots, (x_n, A_n)\}$}  
\end{prooftree}


\caption{Typisierung von Deklarationen}
\label{fig:lama_decls_types}
\end{figure}


%%% nodes
\begin{figure}[H]

% invariant
\begin{prooftree}
\AxiomC{$\Sigma,\Delta,\Gamma \vdash e : \ty{bool}$}
\RightLabel{(invariant)}
\UnaryInfC{$\Sigma,\Delta,\Gamma \vdash
  \syn{invariant } e \syn{ ;} : \ty{ok}$}
\end{prooftree}

TODO
  
\caption{Typisierung von Knoten}
\label{fig:lama_node_types}
\end{figure}

\begin{figure}[H]

\begin{tabular}{ll}
% equality
\AxiomC{}
\RightLabel{(eq)}
\UnaryInfC{$\vdash
  \syn{=} : \forall t : Type. \; t \Rightarrow t \Rightarrow bool$}
\DisplayProof
\quad &
% relations
\AxiomC{}
\RightLabel{(rels)}
\UnaryInfC{$\vdash
  \boxempty : \forall t : Num. \; t \Rightarrow t \Rightarrow bool$}
\DisplayProof \\
& $\boxempty \in \{\syn{<}, \syn{<=}, \syn{>}, \syn{>=}\}$ \\
\\

% boolean functions
% unary boolean
\AxiomC{}
\RightLabel{(unary-bool)}
\UnaryInfC{$\vdash
  \syn{not} : \ty{bool} \Rightarrow \ty{bool}$}
\DisplayProof
&
% binary boolean
\AxiomC{}
\RightLabel{(bin-bool)}
\UnaryInfC{$\vdash
  \boxempty : \ty{bool} \Rightarrow \ty{bool}
  \Rightarrow \ty{bool}$}
\DisplayProof \\
& $\boxempty \in \{\syn{and}, \syn{or}, \syn{xor}, \syn{=>}\}$ \\
\end{tabular}

% arithmetic
% general arithmetic
%<*arith>
\begin{prooftree}
\AxiomC{}
\RightLabel{(arith), $ \boxempty \in \{\syn{+}, \syn{-}, \syn{*}\}$}
\UnaryInfC{$\vdash
  \boxempty : \forall t : Num. \; t \Rightarrow t \Rightarrow t$}
\end{prooftree}
%</arith>

% integer arithmetic
%<*intarith>
\begin{prooftree}
\AxiomC{}
\RightLabel{(int-arith), $\boxempty \in \{\syn{div}, \syn{mod}\}$}
\UnaryInfC{$\vdash
  \boxempty : \ty{int} \Rightarrow \ty{int} \Rightarrow \ty{int}$}
\end{prooftree}
%</intarith>

\begin{prooftree}
\AxiomC{}
\RightLabel{(real-arith)}
\UnaryInfC{$\vdash
  \syn{/} : \ty{real} \Rightarrow \ty{real} \Rightarrow \ty{real}$}
\end{prooftree}

\begin{prooftree}
\AxiomC{}
\RightLabel{(ite)}
\UnaryInfC{$\vdash
  \syn{ite} : \forall t : Type. \;
  \ty{bool} \Rightarrow t \Rightarrow t \Rightarrow t$}
\end{prooftree}
  
\caption{Typisierung von Relationen und Operatoren}
\label{fig:lama_rel_arith_types}
\end{figure}

%%% universes
\begin{figure}[H]

% universe instances
\begin{tabular}{ll}
\AxiomC{}
\RightLabel{(num-univ)}
\UnaryInfC{$\vdash T : Num$}
\DisplayProof
\quad &
\AxiomC{}
\RightLabel{(bool-univ)}
\UnaryInfC{$\ty{bool} : Type$}
\DisplayProof \\
$T \in \{\ty{int}, \ty{real}, \ty{sint[n]}, \ty{uint[n]}\},
n \in \N $ & \\
\\
\AxiomC{$x \in \Sigma$}
\RightLabel{(enum-univ)}
\UnaryInfC{$\Sigma \vdash \ty{x} : Type$}
\DisplayProof
\quad &
\AxiomC{$T_1 : U_1 \, \ldots \, T_n : U_n$}
\RightLabel{(prod-univ)}
\UnaryInfC{$\vdash \ty{(\#} \; T_1 \, \ldots \, T_n \ty{)} : Type$}
\DisplayProof
\end{tabular}

% universe generalisation
%<*univgen>
\begin{prooftree}
\AxiomC{$\vdash T : Num$}
\RightLabel{(univ-gen)}
\UnaryInfC{$\vdash T : Type$}
\end{prooftree}
%</univgen>

\caption{Typ-Universen}
\label{fig:lama_type_universes}
\end{figure}

%%% quantifier elimination and function application
\begin{figure}[H]

% quantifier elimination
%<*forallelim>
\begin{prooftree}
\AxiomC{$\Sigma,\Delta,\Gamma \vdash M : (\forall t : U. \; A)$}
\AxiomC{$\Sigma,\Delta,\Gamma \vdash T : U$}
\RightLabel{($\forall$-E)}
\BinaryInfC{$\Sigma,\Delta,\Gamma \vdash M : A[T/t]$}  
\end{prooftree}
%</forallelim>

% application
%<*funapp>
\begin{prooftree}
\AxiomC{$\Sigma,\Delta,\Gamma \vdash M : A \Rightarrow B$}
\AxiomC{$\Sigma,\Delta,\Gamma \vdash N : B$}
\RightLabel{(fun-app)}
\BinaryInfC{$\Sigma,\Delta,\Gamma \vdash
  \syn{(} M \; N \syn{)} : B$}
\end{prooftree}
%</funapp>

\caption{Interne Quantoren- und Applikationsregeln}
\label{fig:lama_quantifier_app}
\end{figure}

%%% product expressions
\begin{figure}[H]

% product introduction
\begin{prooftree}
\AxiomC{$\Sigma,\Delta,\Gamma \vdash
  e_1 : A_1, \, \ldots \, , e_n : A_n$}
\RightLabel{(\#-intro)}
\UnaryInfC{$\Sigma,\Delta,\Gamma \vdash
  \syn{(\# } e_1, \, \ldots \, , e_n \syn{)}
  : \syn{(\# } A_1, \, \ldots \, , A_n \syn{)}$}
\end{prooftree}

% product elimination
\begin{prooftree}
\AxiomC{$\Sigma,\Delta,\Gamma \vdash
  x : \syn{(\# } A_1, \, \ldots \, , A_n \syn{)}$}
\RightLabel{(\#-elim)}
\UnaryInfC{$\Sigma,\Delta,\Gamma \vdash
  \syn{(project } x \; i) : A_i$}
\end{prooftree}


\label{fig:lama_prod_expr_types}
\caption{Typisierung von Produkten}
\end{figure}

%</typerules>

\end{document}
